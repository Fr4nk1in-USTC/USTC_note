% Copyright (c) 2022 Fr4nk1in-USTC
% 
% This software is released under the MIT License.
% https://opensource.org/licenses/MIT
\documentclass[
    mode=hazy,
    color=blue,
    device=normal,
    lang=cn
]{elegantnote}

% Title Format
\title{第 0 章 预备知识}
\author{Fr4nk1in-USTC}
\institute{中国科学技术大学计算机学院}
\date{\zhtoday}

% Packages
\usepackage{amsmath,amsthm,amssymb,amsfonts,amscd}
\newtheorem{axiom}{公理}[section]

\begin{document}
    \maketitle
    \section{集论初等概念}
    \begin{itemize}
        \item 子集与包含关系 $\subseteq$
        $$
        A\subseteq B \Leftrightarrow \forall x\in A, x\in B
        $$
        \item 集合相等 $=$
        $$
        A=B \Leftrightarrow A\subseteq B\ \text{且}\ B\subseteq A
        $$
        \item 幂集 $\mathcal{P(\cdot)}$
        $$
        \mathcal{P}(A) = \{a\ \vert\ a\subseteq A\}
        $$
        \item 集合运算 $\cup$ $\cap$ $-$
        \begin{itemize}
            \item 并集 $\cup$
            $$
            A\cup B = \{x\ \vert\ x\in A\ \text{或}\ x \in B\}
            $$
            \item 交集 $\cap$
            $$
            A\cap B = \{x\ \vert\ x\in A\ \text{且}\ x \in B\}
            $$
            作为集的运算, 并和交都满足交换律, 结合律和分配律.
            \item 差集 $-$
            $$
            A-B=\{x\ \vert x\in A\ \text{且}\ x \notin B\}
            $$
        \end{itemize}
        \item 积集 $\times$
        $$
        A\times B = \{(a,b)\ \vert\ a\in A,b \in B\}
        $$
        $$
        \prod_{i=1}^n A_i = A_1\times \cdots \times A_n = \{(a_1, \cdots, a_n)\ \vert\ a_1\in A_1,\cdots, a_n\in A_n\}
        $$
        $$
        A^0 = \varnothing  , A^1 = A, A^n = \prod_{i = 1}^n A
        $$
        \item 关系
        \begin{itemize}
            \item $A$ 到 $B$ 的关系 $R$ : $R\subseteq A\times B$
            \item $A$ 上的 $n$ 元关系 $R$ : $R\subseteq A^n$
        \end{itemize}
        \item 等价关系
        \begin{itemize}
            \item $A$ 上的等价关系 $R$ : 满足以下三条性质的二元关系 $R(\subseteq A^2)$
            \begin{enumerate}[label=$\arabic*^\circ$]
                \item 自反性: $\forall x\in A, (x,x)\in R$
                \item 对称性: $\forall x,y\in A, (x,y)\in R\Leftrightarrow (y,x)\in R$
                \item 可递性: $\forall x,y\in A, (x,y), (y,z)\in R\Rightarrow (x,z)\in R$
            \end{enumerate}
            若 $(a,b)\in R$ 则称 $a$ 与 $b$ 等价, 记作 $a\sim b$.
            \item 等价类
            
            A 中与 $a(\in A)$ 等价的所有元素形成的集叫做由 $a$ 形成的 $R$ 等价类, 记作

            $$
            [a] = \{x\ \vert\ x\in A, x\sim a\}
            $$

            不同的等价类之间没有公共元素, 所以 $A$ 上的任何等价关系 $R$ 都确定了 $A$ 的一个分类.

            \item 商集: 设 $R$ 是 $A$ 上的等价关系, 所有 $R$ 等价类的集叫做商集, 记作 $A/R$.
        \end{itemize}
        \item 映射: 一种特殊的关系
        \begin{description}
            \item[定义] 设 $f$ 是集 $X$ 到集 $Y$ 的一个关系 (即 $f\subseteq X\times Y$), 且对任意 $ x\in X$ 都有且只有一个 $y\in Y$ 使得 $(x,y)\in f$, 那么我们称 $f$ 是从 $X$ 到 $Y$ 的函数或映射, 记作 $f:X\to Y$.
            \item[象与原象] 若 $(x_0,y_0)\in f$, 那么我们称 $y_0$ 为 $x_0$ 的象, $x_0$ 是 $y_0$ 的原象, 记作 $x_0\mapsto y_0$ 或 $y_0=f(x_0)$.
            \item[定义域与值域] $X$ 叫做 $f$ 的定义域. $X$ 中元素在 $Y$ 中的象的全体是 $Y$ 的一个子集, 叫做 $f$ 的值域.
            \item[满射] 映射 $f:X\to Y$ 的值域就是 $Y$.
            \item[单射] 映射 $f:X\to Y$ 满足对任意的 $x_1, x_2\in X$, 有
            $$
            x_1\neq x_2\Rightarrow f(x_1)\neq f(x_2)
            $$
            \item[双射] 映射 $f:X\to Y$ 既是单射又是满射. 此时称 $X$ 和 $Y$ 之间存在一一对应, 或者称 $X$ 和 $Y$ 等势, 也称 $X$ 和 $Y$ 有相同的基数.
            \begin{itemize}
                \item 双射 $f:X\to Y$ 的逆映射 $f^{-1}:Y\to X$ 也是双射. ($f(x)=y\Leftrightarrow f^{-1}(y) = x$)
                \item 双射 $f:X\to Y$ 与双射 $g:Y\to Z$ 的复合映射 $g\circ f:X\to Z$ 也是双射. ($(g\circ f)(x)=g(f(x))$)
            \end{itemize} 
        \end{description}
        \item $n$ 元运算: 集 $A$ 上的 $n$ 元函数 $f:A^n\to A$ 叫做 $A$ 上的 $n$ 元运算.
    \end{itemize}
    \section{Peano 自然数公理}
    我们把自然数集 $\mathbb{N}$ 看成是满足以下五条公理的集.
    \begin{axiom}
        $0\in\mathbb{N}$.
    \end{axiom}
    \begin{axiom}
        若 $x\in \mathbb{N}$, 则 $x$ 有且只有一个后继 $x'\in\mathbb{N}$.
    \end{axiom}
    \begin{axiom}
        对任意 $x\in\mathbb{N}$, $x'\neq0$.
    \end{axiom}
    \begin{axiom}
        对任意 $x_1,x_2\in \mathbb{N}$, 若 $x_1\neq x_2$, 则 $x_1'\neq x_2'$.
    \end{axiom}
    \begin{axiom}
        设 $M\subseteq \mathbb{N}$. 若 $0\in M$, 且当 $x\in M$ 时也有 $x'\in M$, 则 $M=\mathbb{N}$.
    \end{axiom}
    有下面的常用结论.
    \begin{theorem}[强归纳法]
        假设与自然数 $n$ 有关的命题 $P(n)$ 满足以下两个条件:
        \begin{enumerate}[label=$\arabic*^\circ$]
            \item $P(0)$ 成立;
            \item 对于 $m>0$, 若 $k<m$ 时 $P(k)$ 都成立, 则 $P(m)$ 也成立,
        \end{enumerate}
        则 $P(n)$ 对所有的自然数 $n$ 都成立.
    \end{theorem}
    \begin{proof}
        只要证明集合 $S=\{n\ \vert\ P(n)\text{ 不成立}\}$ 为空集即可, 使用反证法, 略.
    \end{proof}
    \section{可数集}
    \begin{definition}
        有限集是指空集或与 $\{0,1,\cdots,n\}, n\in \mathbb{N}$ 等势的集.
        可数集是指与自然数集 $\mathbb{N}$ 等势的集, $\mathbb{N}$ 显然也是可数集.
    \end{definition}
    \begin{proposition}
        可数集的无限子集也是可数集.
    \end{proposition}
    \begin{proposition}
        若存在无限集 $B$ 到可数集 $A$ 的单射, 则 $B$ 为可数集.
    \end{proposition}
    \begin{proposition}
        \begin{enumerate}[label=$\arabic*^\circ$]
            \item 若 $A$ 可数且 $B$ 非空有限或可数, 则 $A\times B$ 和 $B\times A$ 都可数.
            \item 若 $A_1, \cdots, A_n$ 中至少有一个可数而其他为非空有限或可数, 则 $\prod_{i = 1}^n A_i$ 可数.
        \end{enumerate}
    \end{proposition}
    \begin{proposition}
        \begin{enumerate}[label=$\arabic*^\circ$]
            \item 若 $A$ 可数且 $B$ 有限或可数, 则 $A\cup B$ 也可数.
            \item 若 $A_1, \cdots, A_n$ 中至少有一个可数而其他为有限或可数, 则 $\bigcup_{i = 1}^n A_i$ 可数.
        \end{enumerate}
    \end{proposition}
    \begin{proposition}
        若 $A$ 可数, 则 $\bigcup_{n=1}^{\infty}A^n$ 可数.
    \end{proposition}
    \begin{proposition}
        若 $A$ 可数, 则所有由 $A$ 的元素形成的有限序列构成的集 $B$ 也可数.
    \end{proposition}
    \begin{proposition}
        若每个 $A_i$ 有限或可数, 且 $\bigcup_{i\in\mathbb{N}}A_i$ 是无限集, 则 $\bigcup_{i\in\mathbb{N}}A_i$ 可数.
    \end{proposition}
    根据下面的 Cantor 定理, 存在大量的不可数的无限集.
    \begin{theorem}
        集 $A$ 和 $A$ 的幂集 $\mathcal{P}(A)$ 不等势. 
    \end{theorem}
    说明可数集的幂集是不可数的. 
    \begin{example}
        实数集 $\mathbb{R}$, 区间 $(0,1)$ 都是不可数的.
    \end{example}
\end{document}